\documentclass[twoside]{article}

% Fonts + Colors ~~~~

\usepackage{lipsum}
\usepackage{fontspec}
\usepackage{dashrule}
\usepackage[export]{adjustbox}
\usepackage[usenames, dvipsnames]{color}

\definecolor{colorTitle}{RGB}{200, 77, 179}  % Pink
\definecolor{colorSection}{RGB}{102, 153, 255}  % Blue
\definecolor{colorFooter}{RGB}{120, 120, 120}  % Gray
\definecolor{LightPink}{RGB}{200, 30, 38}
\definecolor{LightBlue}{RGB}{165, 195, 255}
\definecolor{LightGray}{RGB}{220, 220, 220}

\newfontfamily\Avenir[
    Path= fonts/,
    Extension= .otf,
    UprightFont = *-45Book,
    ItalicFont = *-45BookOblique,
    BoldFont = *-95Black,
]{AvenirLTStd}

\newfontfamily\TradeGothic[
    Path = fonts/,
    Extension = .otf,
    UprightFont = *-No18,
    BoldFont = *-No20Bold,
]{TradeGothicCondensed}

\setmainfont[Ligatures=TeX]{Avenir}
\Avenir\fontsize{9pt}{12pt}\selectfont


% Formatting ~~~~

\usepackage{geometry}

\geometry{
    letterpaper,
    top=1.75in,
    bottom=1.75in,
    inner=1in,
    outer=1.5in,
    headsep=1in,
    headheight=0.5in,
    footskip=1.2in,
    % showframe
}

\usepackage[document]{ragged2e}
\parskip=12pt
\parindent=0pt

\usepackage{sectsty}
\sectionfont{\color{colorSection}\mdseries\uppercase}
\subsectionfont{\color{Periwinkle}}


% Headers + Footers ~~~~

\usepackage{fancyhdr}
\pagestyle{fancy}

\renewcommand{\sectionmark}[1]{\markboth{#1}{}}
\usepackage{fourier-orns}

\fancyhead{} % clear all header fields
\fancyfoot{} % clear all footer fields
\fancyfoot[LE]{\thepage \hspace{0.4cm} \TradeGothic\selectfont\textcolor{colorTitle}{\uppercase{Oh, The Places They'll Go}}}
\fancyfoot[RO]{\TradeGothic\selectfont\textcolor{colorFooter}{\leftmark} \hspace{0.4cm} \thepage}
\renewcommand{\headrulewidth}{0pt}

\fancypagestyle{tablepage}{  % No Header, Regular Footer
    \fancyhead{}
    \fancyfoot{}
    \renewcommand\headrule{\hrulefill
\raisebox{-2.1pt}[10pt][10pt]{\quad\decofourleft\decotwo\decofourright\quad}\hrulefill}
}


% Links + Lists ~~~~

\PassOptionsToPackage{hyphens}{url}\usepackage[colorlinks=true, linkcolor=cyan, urlcolor=blue, citecolor=blue]{hyperref}
\newcommand{\superscript}[1]{$^{#1}$}

\usepackage[super,numbers]{natbib}

\makeatletter
\renewcommand\@biblabel[1]{\superscript{#1}}
\makeatother

\renewcommand{\refname}{\vspace{-0.65cm}}

\usepackage{enumitem}
\setlist{nosep, itemsep=0pt, parsep=0pt, before={\TradeGothic\large\color{Gray}}}

\renewcommand{\labelitemi}{\color{colorTitle}$\triangleright$}
\renewcommand{\labelitemii}{\color{Periwinkle}$\cdot$}
\renewcommand\labelitemiii{$\circ$}


% Block Formats ~~~~

\usepackage{mdframed}

\newenvironment{quote-block}{  % quote block
    \begin{quote}
    \TradeGothic\selectfont
    \color{colorSection}
    \begin{large}
}
{
    \end{large}
    \end{quote}
}

\newenvironment{odd-block}{  % text block for odd-numbered pages
    \vspace{0.5cm}
    \begin{mdframed}[backgroundcolor=LightBlue, linecolor=LightBlue, userdefinedwidth=8in, leftmargin=0cm, innerleftmargin=1cm, innerrightmargin=6cm, innertopmargin=0.5cm, innerbottommargin=1cm]
    \color{white}
}
{
    \end{mdframed}
    \vspace{0.5cm}
}


% Figures ~~~~

\usepackage{graphicx}
\usepackage{caption}

\DeclareCaptionFont{tg}{\TradeGothic\selectfont}
\DeclareCaptionFont{colorTitle}{\color{colorTitle}}

\captionsetup{labelfont={tg, colorTitle}, figurename=FIGURE}

\newcommand{\addFigure}[4][0.75] {  % Add single figure
\begin{figure}[!ht]
    \vspace{0.45cm}
    \centering

    \includegraphics[width=#1\textwidth]{images/#2}
    \caption{\TradeGothic\selectfont\textcolor{colorFooter}{\uppercase{#3}}}
    \label{fig:#4}

\end{figure}
}

\newcommand{\addFigureSet}[3] {  % Add 2x2 figure set
\begin{figure}[!ht]
    \vspace{0.45cm}
    \centering

    \includegraphics[width=0.45\textwidth, cfbox=LightGray 1pt]{images/#1/visit_count}\hfill
    \includegraphics[width=0.45\textwidth, cfbox=LightGray 1pt]{images/#1/timeline}

    \smallskip

    \includegraphics[width=0.45\textwidth, cfbox=LightGray 1pt]{images/#1/median_income}\hfill
    \includegraphics[width=0.45\textwidth, cfbox=LightGray 1pt]{images/#1/income_band}

    \caption{\TradeGothic\selectfont\textcolor{colorFooter}{\uppercase{#2}}}
    \label{fig:#3}

    \vspace{0.45cm}
\end{figure}
}

\newcommand{\addFigureCompare}[2] {  % Add row of 4 figures
\begin{figure}[!h]
    \vspace{0.45cm}
    \hspace{-1.6in}
    \color{Gray}\hdashrule{1.5\textwidth}{1pt}{1pt}\vspace{0.5cm}

    \centering

    \makebox[0.79\paperwidth][c]{\hspace{-3cm}\includegraphics[width=0.2\paperwidth, cfbox=LightGray 1pt]{images/139959/titled_map}\hfill
    \includegraphics[width=0.2\paperwidth, cfbox=LightGray 1pt]{images/110635/titled_map}\hfill
    \includegraphics[width=0.2\paperwidth, cfbox=LightGray 1pt]{images/215293/titled_map}\hfill
    \includegraphics[width=0.2\paperwidth, cfbox=LightGray 1pt]{images/196097/titled_map}}

    \caption{\TradeGothic\selectfont\textcolor{colorFooter}{\uppercase{#1}}}
    \label{fig:#2}
    \makebox[0.79\paperwidth][c]{\hdashrule{1.5\paperwidth}{1pt}{1pt}}

\end{figure}
\vspace{0.45cm}
}


\begin{document}

% Title Page ~~~~

\newgeometry{top=0cm, bottom=0cm, left=0cm, right=0cm}
\begin{titlepage}

    \begin{figure}
        \includegraphics[width=.25\textwidth, right]{images/emra.jpg}
        \vspace{-1.1cm}
    \end{figure}

    \begin{mdframed}[backgroundcolor=LightBlue, linecolor=LightBlue, innerbottommargin=1.2cm]

        \centering\color{white}\fontsize{30}{60}

        \vspace{3cm}
        \hdashrule{0.75\textwidth}{1pt}{1pt} \\

        \vspace{0.8cm}
        \textbf{\color{LightPink}{OH, THE PLACES THEY'LL GO}} \\

        \vspace{0.6cm}
        \large\textit{Similarities and Differences in Off-Campus Recruiting Strategies \\ of Public Research Universities} \\

        \vspace{0.4cm}
        \hdashrule{0.75\textwidth}{1pt}{1pt} \\

        \vspace{10cm}
        \textbf{Ozan Jaquette} \\
        University of California, Los Angeles \\~\\
        \textit{\color{Gray}{Oct 2018}}

    \end{mdframed}
\end{titlepage}

% Blank Page ~~~~
 
\pagecolor{LightBlue}
\begin{titlepage}
    \color{LightBlue}{x}
\end{titlepage}

% Begin Template ~~~~

\pagecolor{white}
\restoregeometry
\setcounter{page}{1}

\section*{Introduction\markboth{Introduction}{}}  % text inside \markboth{} is displayed in footer

The University of Alabama-Tuscaloosa exemplifies that transformation from state flagship university to the out-of-state flagship.  Resident freshmen increased from 2,028 in 2002-03 to 3,221 in 2008-09 but declined to 2,412 by 2016-17. By contrast, nonresident freshmen increased dramatically from 626 in 2002-03 to 1,895 in 2008-09 and to 5,147 by 2016-17.  This period was also witnessed the erosion of state appropriations, which increased from  \$165 million in 2002-03 to \$227 million in 2007-08, but declined sharply to \$149 million by 2010-11 following the Great Recession, increasing only modestly to \$153 million by 2015-16 (XXX CPI).  By contrast, driving by nonresident enrollment growth, net tuition revenue increased dramatically, from \$102 million in 2002-03 to \$220 million by 2007-08 to \$492 million by 2015-16.

Nonresident enrollment growth at the University of Alabama also coincided with declining socioeconomic and racial diversity.  The percent of full-time freshman receiving Pell Grants declined from 21.2\% in 2010-11 to 17.8\% in 2015-16.  Additionally, while the percent of 18-24 year-olds in Alabama who identify as Black increased from 31.4\% in 2010-11 to 32.7\% in 2016-17, the percent of full-time freshman at the University of Alabama who identify as Black declined from 11.8\% to 8.0\% over the same time period.

This transformation in tuition revenue and student composition did not result from sudden, unexpected shifts in student demand. Rather, the University of Alabama developed arguably the most sophisticated and extensive approach to student recruiting in public higher education.  Utilizing the ``data science'' expertise of enrollment management consulting firms, the university identifies desirable ``prospects'' and plies these prospects with a targeted cocktail of emails, brochures, paid advertising (e.g., pay-per-click ads from Google), off-campus recruiting visits to ``feeder'' high schools, and a savvy social media campaign. 

This report focuses on off-campus recruiting visits (e.g., visits to local high schools, community colleges, hotel receptions).  In 2017 alone, University of Alabama admissions representatives made 4,261 off-campus recruiting visits.  However, only 382 of these visits occurred in Alabama.  Further, the University visited only 32\% of Alabama public high schools. These in-state public high school visits were concentrated relatively, affluent, predominantly White communities, largely avoiding high schools in Alabama's ``Black Belt,'' which enroll the largest concentration of African American Students.  However, these in-state recruiting efforts were dwarfed by the 3,879 out-of-state recruiting visits, which spanned metropolitan areas across the U.S. The University made 2,252 visits to out-of-state public high schools. These visits focused on schools in affluent communities, with visited schools having a much higher percent of White students than non-visited schools.  Incredibly, the University made 914 visits to out-of-state private high schools, more than double the total number of in-state recruiting visits.

The University of Alabama represents an extreme case of a transformation occurring at many, but not all, public research universities across the nation.  Public research universities were founded to provide upward mobility for high-achieving state residents. Designated the unique responsibility of preparing the the future professional, business, and civic leaders of the state, public research universities provided -- quoting the 19th Century University of Michigan President James Angell -- ``an uncommon education for the common man'' who could not afford tuition at elite private institutions.  Unfortunately, public research universities increasingly an enroll an affluent student body that is unrepresentative of the socioeconomic and racial diversity of the states they serve, raising concerns that these engines of opportunity have become engines of inequality [CITE/QUOTE].

Contemporary policy debates about racial and socioeconomic inequality in college access tend to focus on the ``achievement gap'' and on ``undermatching,'' the idea that high-achieving, low-income students fail to apply to good colleges because they have bad guidance at home and at school.  These explanations focus on ``deficiencies'' of students and K-12 schools. As such, state and national policy interventions to increase college access mostly focus on student academic achievement and decision-making [CITE]. Affordability is another barrier to access. In recent decades, particularly following the Great Recession of 2008, states disinvested in public universities, and these state budget cuts have been associated with steep rises in tuition price. 

Public research universities position themselves as progressive actors that remain committed to the access mission despite state funding cuts and despite the deficiencies of students and K-12 schools.  Universities point to the adoption of policies such as holistic admissions, need-based financial aid, and outreach/pipeline programs as evidence of their commitment to access [CITE].  However, many public research universities have dramatically increased nonresident enrollment [CITE] and many universities have adopted financial aid policies that specifically target non-resident students with mediocre academic achievement [CITE]. Meanwhile, many high-achieving, low-income students attend community colleges [CITE], which dramatically lower the probability of obtaining a BA

These enrollment trends suggest an alternative explanation for growing racial and socioeconomic inequality in access to public  research universities: university enrollment priorities privilege affluent students and are biased against low-income students and communities of color.  While this explanation conflicts with university public relations rhetoric and the slew of access-oriented policies adopted, decades of research on organizational behavior shows that formal policy adoption (e.g., outreach, financial aid programs) is often a symbolic effort to appease external stakeholders rather than a substantive effort to solve the problem.  We argue that knowing which student populations are actually targeted by university recruiting efforts is a more credible indicator of enrollment priorities.  Unfortunately, research on recruiting is rare because data on university recruiting behavior are difficult to obtain. 

This study does XXXXX.  If university enrollment priorities are biased against low-income students and communities of color, then policies solutions that focus solely on students and K-12 schools will not overcome access inequality. ADD BRIEF TEXT THAT PREVIOUS OUTLINE, FINDINGS, AND HINTS AT POLICY IMPLICATIONS/RECOMMENDATIONS TO FOLLOW



%Meanwhile, the percent of full-time freshman receiving Pell Grants stagnated, increasing from 17.9\% in 2002-03 to a high of 21.2\% in 2010-11, following the massive increase in federal Pell funding by the Obama Administration, but declining to 17.9\% by 2015-16. 


\section*{The Enrollment Management Industry\markboth{The Enrollment Management Industry}{}}  % text inside \markboth{} is displayed in footer

\lipsum[1]

% Params: width (default: 0.75\textwidth), filename in the images/ folder, caption, fig reference (e.g., ~\ref{fig:funnel})
\addFigure[0.55]{funnel_alt.png}{The Enrollment Funnel.}{funnel}

\subsection*{Research and policy on access inequality}

\lipsum[1]

\begin{quote-block}  % quote block
    \lipsum[2]
\end{quote-block}

\lipsum[3]

\subsection*{Scholarship on recruiting}

\lipsum[1-2]

\begin{odd-block}  % text block for odd-numbered pages
  \vspace{-0.3cm}
  \subsection*{\color{colorTitle}{Key Points}}
  \vspace{-0.2cm}
  \lipsum[1-2]
\end{odd-block}

\lipsum[3]

\section*{Deep-Dive Results\markboth{Deep-Dive Results}{}}

\subsection*{University of Georgia}
\lipsum[4]
% Params: univ_id (which is also the name of the folder inside images/ folder, where the images are located), caption, fig reference
\addFigureSet{139959}{University of Georgia result set.}{uga}
\lipsum[5]

\subsection*{University of California, Berkeley}
\lipsum[4]
\addFigureSet{110635}{UC Berkeley result set.}{ucberkeley}
\lipsum[4]

\subsection*{University of Pittsburgh}
\lipsum[4]
\addFigureSet{215293}{University of Pittsburgh result set.}{upitt}
\lipsum[5]

\subsection*{Stony Brook}
\lipsum[4]
\addFigureSet{196097}{Stony Brook result set.}{stonybrook}
\lipsum[5]

\section*{Cross-University Results\markboth{Cross-University Results}{}}

\lipsum[1]

\addFigureCompare{Comparison Across Universities\hspace{1cm}}{compare}

\lipsum[6]

\section*{Conclusions\markboth{Conclusions}{}}

\lipsum[1]

\begin{itemize}  % List
	\item Un
	\begin{itemize}
		\item Lorem ipsum dolor sit amet, consectetuer adipiscing elit.
		\item Cras nec ante. Pellentesque a nulla.
	\end{itemize}
	\item Deux
	\begin{itemize}
		\item Nam dui ligula, fringilla a, euismod sodales, sollicitudin vel, wisi.
	\end{itemize}
	\item Ttrois
	\begin{itemize}
		\item Duis aute irure dolor in reprehenderit in voluptate velit esse cillum dolore eu fugiat nulla pariatur.
	\end{itemize}
	
\end{itemize} 

\lipsum[4]

\newpage

% Endnotes ~~~~

\section*{Endnotes\markboth{Endnotes}{}}

TBD

% Final Page ~~~~

\newpage
\pagecolor{LightBlue}
\begin{titlepage}
    \color{LightBlue}{x}
\end{titlepage}

\end{document}
