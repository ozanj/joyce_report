\documentclass[twoside]{article}

% Fonts + Colors ~~~~

\usepackage{lipsum}
\usepackage{fontspec}
\usepackage{dashrule}
\usepackage[export]{adjustbox}
\usepackage[usenames, dvipsnames]{color}

\definecolor{colorTitle}{RGB}{200, 77, 179}  % Pink
\definecolor{colorSection}{RGB}{102, 153, 255}  % Blue
\definecolor{colorFooter}{RGB}{120, 120, 120}  % Gray
\definecolor{LightPink}{RGB}{200, 30, 38}
\definecolor{LightBlue}{RGB}{165, 195, 255}
\definecolor{LightGray}{RGB}{220, 220, 220}

\newfontfamily\Avenir[
    Path= fonts/,
    Extension= .otf,
    UprightFont = *-45Book,
    ItalicFont = *-45BookOblique,
    BoldFont = *-95Black,
]{AvenirLTStd}

\newfontfamily\TradeGothic[
    Path = fonts/,
    Extension = .otf,
    UprightFont = *-No18,
    BoldFont = *-No20Bold,
]{TradeGothicCondensed}

\setmainfont[Ligatures=TeX]{Avenir}
\Avenir\fontsize{9pt}{12pt}\selectfont


% Formatting ~~~~

\usepackage{geometry}

\geometry{
    letterpaper,
    top=1.75in,
    bottom=1.75in,
    inner=1in,
    outer=1.5in,
    headsep=1in,
    headheight=0.5in,
    footskip=1.2in,
    % showframe
}

\usepackage[document]{ragged2e}
\parskip=12pt
\parindent=0pt

\usepackage{sectsty}
\sectionfont{\color{colorSection}\mdseries\uppercase}
\subsectionfont{\color{Periwinkle}}


% Headers + Footers ~~~~

\usepackage{fancyhdr}
\pagestyle{fancy}

\renewcommand{\sectionmark}[1]{\markboth{#1}{}}
\usepackage{fourier-orns}

\fancyhead{} % clear all header fields
\fancyfoot{} % clear all footer fields
\fancyfoot[LE]{\thepage \hspace{0.4cm} \TradeGothic\selectfont\textcolor{colorTitle}{\uppercase{Oh, The Places They'll Go}}}
\fancyfoot[RO]{\TradeGothic\selectfont\textcolor{colorFooter}{\leftmark} \hspace{0.4cm} \thepage}
\renewcommand{\headrulewidth}{0pt}

\fancypagestyle{tablepage}{  % No Header, Regular Footer
    \fancyhead{}
    \fancyfoot{}
    \renewcommand\headrule{\hrulefill
\raisebox{-2.1pt}[10pt][10pt]{\quad\decofourleft\decotwo\decofourright\quad}\hrulefill}
}


% Links + Lists ~~~~

\PassOptionsToPackage{hyphens}{url}\usepackage[colorlinks=true, linkcolor=cyan, urlcolor=blue, citecolor=blue]{hyperref}
\newcommand{\superscript}[1]{$^{#1}$}

\usepackage[super,numbers]{natbib}

\makeatletter
\renewcommand\@biblabel[1]{\superscript{#1}}
\makeatother

\renewcommand{\refname}{\vspace{-0.65cm}}

\usepackage{enumitem}
\setlist{nosep, itemsep=0pt, parsep=0pt, before={\TradeGothic\large\color{Gray}}}

\renewcommand{\labelitemi}{\color{colorTitle}$\triangleright$}
\renewcommand{\labelitemii}{\color{Periwinkle}$\cdot$}
\renewcommand\labelitemiii{$\circ$}


% Block Formats ~~~~

\usepackage{mdframed}

\newenvironment{quote-block}{  % quote block
    \begin{quote}
    \TradeGothic\selectfont
    \color{colorSection}
    \begin{large}
}
{
    \end{large}
    \end{quote}
}

\newenvironment{odd-block}{  % text block for odd-numbered pages
    \vspace{0.5cm}
    \begin{mdframed}[backgroundcolor=LightBlue, linecolor=LightBlue, userdefinedwidth=8in, leftmargin=0cm, innerleftmargin=1cm, innerrightmargin=6cm, innertopmargin=0.5cm, innerbottommargin=1cm]
    \color{white}
}
{
    \end{mdframed}
    \vspace{0.5cm}
}


% Figures ~~~~

\usepackage{graphicx}
\usepackage{caption}

\DeclareCaptionFont{tg}{\TradeGothic\selectfont}
\DeclareCaptionFont{colorTitle}{\color{colorTitle}}

\captionsetup{labelfont={tg, colorTitle}, figurename=FIGURE}

\newcommand{\addFigure}[4][0.75] {  % Add single figure
\begin{figure}[!ht]
    \vspace{0.45cm}
    \centering

    \includegraphics[width=#1\textwidth]{images/#2}
    \caption{\TradeGothic\selectfont\textcolor{colorFooter}{\uppercase{#3}}}
    \label{fig:#4}

\end{figure}
}

\newcommand{\addFigureSet}[3] {  % Add 2x2 figure set
\begin{figure}[!ht]
    \vspace{0.45cm}
    \centering

    \includegraphics[width=0.45\textwidth, cfbox=LightGray 1pt]{images/#1/visit_count}\hfill
    \includegraphics[width=0.45\textwidth, cfbox=LightGray 1pt]{images/#1/timeline}

    \smallskip

    \includegraphics[width=0.45\textwidth, cfbox=LightGray 1pt]{images/#1/median_income}\hfill
    \includegraphics[width=0.45\textwidth, cfbox=LightGray 1pt]{images/#1/income_band}

    \caption{\TradeGothic\selectfont\textcolor{colorFooter}{\uppercase{#2}}}
    \label{fig:#3}

    \vspace{0.45cm}
\end{figure}
}

\newcommand{\addFigureCompare}[2] {  % Add row of 4 figures
\begin{figure}[!h]
    \vspace{0.45cm}
    \hspace{-1.6in}
    \color{Gray}\hdashrule{1.5\textwidth}{1pt}{1pt}\vspace{0.5cm}

    \centering

    \makebox[0.79\paperwidth][c]{\hspace{-3cm}\includegraphics[width=0.2\paperwidth, cfbox=LightGray 1pt]{images/139959/titled_map}\hfill
    \includegraphics[width=0.2\paperwidth, cfbox=LightGray 1pt]{images/110635/titled_map}\hfill
    \includegraphics[width=0.2\paperwidth, cfbox=LightGray 1pt]{images/215293/titled_map}\hfill
    \includegraphics[width=0.2\paperwidth, cfbox=LightGray 1pt]{images/196097/titled_map}}

    \caption{\TradeGothic\selectfont\textcolor{colorFooter}{\uppercase{#1}}}
    \label{fig:#2}
    \makebox[0.79\paperwidth][c]{\hdashrule{1.5\paperwidth}{1pt}{1pt}}

\end{figure}
\vspace{0.45cm}
}


\begin{document}

% Title Page ~~~~

\newgeometry{top=0cm, bottom=0cm, left=0cm, right=0cm}
\begin{titlepage}

    \begin{figure}
        \includegraphics[width=.25\textwidth, right]{images/emra.jpg}
        \vspace{-1.1cm}
    \end{figure}

    \begin{mdframed}[backgroundcolor=LightBlue, linecolor=LightBlue, innerbottommargin=1.2cm]

        \centering\color{white}\fontsize{30}{60}

        \vspace{3cm}
        \hdashrule{0.75\textwidth}{1pt}{1pt} \\

        \vspace{0.8cm}
        \textbf{\color{LightPink}{OH, THE PLACES THEY'LL GO}} \\

        \vspace{0.6cm}
        \large\textit{Similarities and Differences in Off-Campus Recruiting Strategies \\ of Public Research Universities} \\

        \vspace{0.4cm}
        \hdashrule{0.75\textwidth}{1pt}{1pt} \\

        \vspace{10cm}
        \textbf{Ozan Jaquette} \\
        University of California, Los Angeles \\~\\
        \textit{\color{Gray}{Oct 2018}}

    \end{mdframed}
\end{titlepage}

% Blank Page ~~~~
 
\pagecolor{LightBlue}
\begin{titlepage}
    \color{LightBlue}{x}
\end{titlepage}

% Begin Template ~~~~

\pagecolor{white}
\restoregeometry
\setcounter{page}{1}

\section*{Introduction\markboth{Introduction}{}}  % text inside \markboth{} is displayed in footer

The University of Alabama-Tuscaloosa exemplifies that transformation from state flagship university to the out-of-state flagship.  Resident freshmen increased from 2,028 in 2002-03 to 3,221 in 2008-09 but declined to 2,412 by 2016-17. By contrast, nonresident freshmen increased dramatically from 626 in 2002-03 to 1,895 in 2008-09 and to 5,147 by 2016-17.  This period was also witnessed the erosion of state appropriations, which increased from  \$165 million in 2002-03 to \$227 million in 2007-08, but declined sharply to \$149 million by 2010-11 following the Great Recession, increasing only modestly to \$153 million by 2015-16 (XXX CPI).  By contrast, driving by nonresident enrollment growth, net tuition revenue increased dramatically, from \$102 million in 2002-03 to \$220 million by 2007-08 to \$492 million by 2015-16.

Nonresident enrollment growth at the University of Alabama also coincided with declining socioeconomic and racial diversity.  The percent of full-time freshman receiving Pell Grants declined from 21.2\% in 2010-11 to 17.8\% in 2015-16.  Additionally, while the percent of 18-24 year-olds in Alabama who identify as Black increased from 31.4\% in 2010-11 to 32.7\% in 2016-17, the percent of full-time freshman at the University of Alabama who identify as Black declined from 11.8\% to 8.0\% over the same time period.

This transformation in tuition revenue and student composition did not result from sudden, unexpected shifts in student demand. Rather, the University of Alabama developed arguably the most sophisticated and extensive approach to student recruiting in public higher education.  Utilizing the ``data science'' expertise of enrollment management consulting firms, the university identifies desirable ``prospects'' and plies these prospects with a targeted cocktail of emails, brochures, paid advertising (e.g., pay-per-click ads from Google), off-campus recruiting visits to ``feeder'' high schools, and a savvy social media campaign. 

This report focuses on off-campus recruiting visits (e.g., visits to local high schools, community colleges, hotel receptions).  In 2017 alone, University of Alabama admissions representatives made 4,261 off-campus recruiting visits.  However, only 382 of these visits occurred in Alabama.  Further, the University visited only 32\% of Alabama public high schools. These in-state public high school visits were concentrated relatively, affluent, predominantly White communities, largely avoiding high schools in Alabama's ``Black Belt,'' which enroll the largest concentration of African American Students.  However, these in-state recruiting efforts were dwarfed by the 3,879 out-of-state recruiting visits, which spanned metropolitan areas across the U.S. The University made 2,252 visits to out-of-state public high schools. These visits focused on schools in affluent communities, with visited schools having a much higher percent of White students than non-visited schools.  Incredibly, the University made 914 visits to out-of-state private high schools, more than double the total number of in-state recruiting visits.

The University of Alabama represents an extreme case of a transformation occurring at many, but not all, public research universities across the nation.  Public research universities were founded to provide upward mobility for high-achieving state residents. Designated the unique responsibility of preparing the the future professional, business, and civic leaders of the state, public research universities provided -- quoting the 19th Century University of Michigan President James Angell -- ``an uncommon education for the common man'' who could not afford tuition at elite private institutions.  Unfortunately, public research universities increasingly an enroll an affluent student body that is unrepresentative of the socioeconomic and racial diversity of the states they serve, raising concerns that these engines of opportunity have become engines of inequality [CITE/QUOTE].

Contemporary policy debates about racial and socioeconomic inequality in college access tend to focus on the ``achievement gap'' and on ``undermatching,'' the idea that high-achieving, low-income students fail to apply to good colleges because they have bad guidance at home and at school.  These explanations focus on ``deficiencies'' of students and K-12 schools. As such, state and national policy interventions to increase college access mostly focus on student academic achievement and decision-making [CITE]. Affordability is another barrier to access. In recent decades, particularly following the Great Recession of 2008, states disinvested in public universities, and these state budget cuts have been associated with steep rises in tuition price. 

Public research universities position themselves as progressive actors that remain committed to the access mission despite state funding cuts and despite the deficiencies of students and K-12 schools.  Universities point to the adoption of policies such as holistic admissions, need-based financial aid, and outreach/pipeline programs as evidence of their commitment to access [CITE].  However, many public research universities have dramatically increased nonresident enrollment [CITE] and many universities have adopted financial aid policies that specifically target non-resident students with modest academic achievement [CITE]. Meanwhile, many high-achieving, low-income students attend community colleges [CITE], which dramatically lower the probability of obtaining a BA

These enrollment trends suggest an alternative explanation for growing racial and socioeconomic inequality in access to public  research universities: university enrollment priorities privilege affluent students and are biased against low-income students and communities of color.  While this explanation conflicts with university public relations rhetoric and the slew of access-oriented policies adopted, decades of research on organizational behavior shows that formal policy adoption (e.g., outreach, financial aid programs) is often a symbolic effort to appease external stakeholders rather than a substantive effort to solve the problem.  We argue that knowing which student populations are actually targeted by university recruiting efforts is a more credible indicator of enrollment priorities.  Unfortunately, research on recruiting is rare because data on university recruiting behavior are difficult to obtain. 

This study does XXXXX.  If university enrollment priorities are biased against low-income students and communities of color, then policies solutions that focus solely on students and K-12 schools will not overcome access inequality. ADD BRIEF TEXT THAT PREVIOUS OUTLINE, FINDINGS, AND HINTS AT POLICY IMPLICATIONS/RECOMMENDATIONS TO FOLLOW



%Meanwhile, the percent of full-time freshman receiving Pell Grants stagnated, increasing from 17.9\% in 2002-03 to a high of 21.2\% in 2010-11, following the massive increase in federal Pell funding by the Obama Administration, but declining to 17.9\% by 2015-16. 


\section*{The Enrollment Management Industry\markboth{The Enrollment Management Industry}{}}  % text inside \markboth{} is displayed in footer

[TRANSITION PARAGRAPH; TOO LONG?] State budget cuts to public research universities are often rationalized on the grounds that organizations can generate their own revenue sources [CITE]. This is often true and tuition revenue is the biggest money-maker for most universities. What policymakers have ignored is that public universities respond to state disinvestment by shifting ``enrollment management'' operations towards the recruitment of students that generate substantial tuition revenue and often by focusing on the pursuit of rankings prestige at the expense of access for state residents.  Therefore, an understanding of how the enrollment management industry works is the first step to understanding the link between state funding policy and university enrollment behaviors.

Enrollment management (EM) is a profession that integrates techniques from marketing and economics in order to ``influence the characteristics and the size of enrolled student bodies'' \citep[p.~xiv]{RN2771}.  EM is also a university administrative structure (e.g., "The Office of Enrollment Management") that coordinates the activities of admissions, financial aid, and marketing and recruiting. 

The broader enrollment management industry consists of professionals working within universities (e.g., vice president for enrollment management, admissions counselors), the associations EM professionals belong to (e.g., National Association for College Admission Counseling), and the marketing and EM consultancies universities hire (e.g., Hobsons, Ruffalo Noel Levitz).

\subsection*{The enrollment funnel}

% Params: width (default: 0.75\textwidth), filename in the images/ folder, caption, fig reference (e.g., ~\ref{fig:funnel})
\addFigure[0.55]{funnel_alt.png}{The Enrollment Funnel.}{emfunnel}

Figure~\ref{fig:emfunnel} depicts the ``enrollment funnel,'' a conceptual tool the industry uses to describe stages in student recruitment in order to inform targeted recruiting interventions.  While scholarship and policy debate focuses on the final stages of the enrollment funnel -- which applicants are admitted \citep[e.g., ][]{RN3536} and financial aid ``leveraging'' to convert admits to enrollees \citep[e.g., ][]{RN1948} -- the EM industry expends substantial resources on earlier stages of the funnel.  ``Prospects'' are ``all the potential students you would want to attract to your institution'' \citep{RN4322}. ``Inquiries'' are prospects that contact the university, including those who respond to initial solicitation by the universities (e.g., email, brochure) and those who reach out on their own (e.g., sending SAT/ACT scores to the university, completing a form on the university admissions website).  Most universities hire EM consulting firms, which utilize sophisticated, data-intensive methodologies, to make recommendations about identifying prospects, soliciting inquiries, converting prospects and inquiries into applicants, etc. For example, from 2010 to XXXX the University of Alabama paid \$2.7 million to the EM consulting firm Hobsons \citep{RN4035}[UPDATE NUMBERS/YEARS].

%Most universities hire EM consulting firms -- which integrate university data (e.g., prospects, applicants, and enrollees from previous years) with their own micro data on schools and communities -- to make recommendations about interventions at each stage of the enrollment funnel (e.g., identifying prospects, soliciting inquiries, converting prospects/inquiries into applicants). 


Universities identify prospects primarily by purchasing ``student lists'' from College Board and ACT. From 2010 to XXXX, the University of Alabama paid \$1.2 million to College Board and XXXX to ACT \citep{RN4035}.  \cite{RN4314} found that the median public university purchased 64,000 names. Student lists contain contact details and background information (demographic, socioeconomic, and academic) about individual prospects. Universities control which prospects are included in a list by selecting on criteria such as zip code, race, academic achievement.  

Once identified, prospects are plied with recruiting interventions aimed at soliciting inquiries and applications \citep{RN4323}. Non face-to-face interventions include email, brochures, and text messages.  Face-to-face interventions include on-campus visits and off-campus visits. Additionally, universities utilize paid advertising (e.g., pay-per-click ads from Google, cookie-driven ads targeting prospects who visit your website) and social media (e.g., Twitter, Instagram, YouTube) as a means of generating inquiries and creating positive ``buzz'' amongst prospects. Given the the rise in ``stealth applicants'' who do not inquire before applying, social media enables universities to tell their story to prospects who do not want to be contacted.

Given the focus of this report, what is the role of off-campus visits in student recruitment? In the admissions world, ``travel season'' refers to the mad dash between Labor Day and Thanksgiving when admissions officers host hotel receptions, college fairs, and visit high schools across the country. Research by both EM consulting firms and by scholars describe off-campus recruiting as a means of simultaneously identifying prospects and connecting with prospects already being targeted through mail/email \citep[e.g., ][]{RN4323,RN4315,RN3519}.  \cite{RN4402} found that off-campus visits were the second highest source of inquiries (after student list purchases) for the median public university, accounting for 19.0\% of inquiries for the median public institution and the third highest source of enrollees (after stealth applicants and on-campus visits), accounting for 16\% of enrollees.
%across the country with the goal of soliciting applications

Additionally, ethnographic research by Mitchell Stevens (\citeyear{RN3519}) -- he worked as regional admissions recruiter for a selective liberal arts college -- found that high school visits enabled the College to maintain warm relationships with high school guidance counselors at feeder schools.  Echoing these findings, \cite{RN4402} found that face-to-face meetings were the most effective means of engaging counselors. \cite{RN3519} states that relationships with counselors were essential because ``the College's reputation and the quality of its applicant pool are dependent upon its connections with high schools nationwide'' \citep[p.~53]{RN3519}.  The College visited the same schools year after year because successful recruiting depends on long-term relationships with high schools. The College tended to visit affluent schools, and private schools in particular, because these schools enroll high-achieving students who can afford tuition and because these schools have the resources and motivation to host a successful visit.  


\cite{RN4324} analyzed high school visits from the student perspective. High school visits influenced where students applied and where they enrolled. The strength of this finding was modest for affluent students with college educated parents, who tended to be more concerned about college prestige and less taken by overtures from colleges. However, this finding was particularly strong for first-generation students and under-represented students of color.  These students often felt that ``school counselors had low expectations for them and were too quick to suggest that they attend community college'' (p. XX) and were drawn to colleges that ``made them feel wanted'' by taking the time to visit them.  Therefore, while \cite{RN4324} shows that college choice for under-served student populations often hinges on which colleges and universities take the time to visit, prior research has not systematically investigated which high schools receive visits by which colleges and universities.

\subsection*{Enrollment goals and recruiting behavior}

While EM industry provides tools for identifying and targeting prospects at each stage of the enrollment funnel, university enrollment priorities dictate which prospects universities actually pursue.  The ``iron triangle'' of enrollment management states that universities pursue the broad enrollment goals of academic profile, revenue, and access \citep{RN2772}. ``Academic profile'' refers to enrolling high-achieving students -- particularly with respect to standardized test scores -- who help the university move up the rankings. ``Revenue'' refers to students who generate high net tuition revenue.  For public universities, the ``access'' goal refers to access for state residents, first-generation students, low-income students, and students of color from historically under-represented racial/ethnic groups. Because resources are scarce, the imagery of the iron triangle suggests that pursuing one goal involves trade-offs with other goals: ``most enrollment management policies [...] do not advance all three objectives; instead they lead to gains in some areas and declines in others'' \citep[p.~221]{RN2772}. Enrollment managers view these trade-offs as an inevitable consequence of organizational enrollment priorities, thereby motivating the question, ``What are the enrollment priorities of public universities?''

Drawing from theories of organizational behavior, we argue that university recruiting behavior is a good indicator of enrollment priorities.  

Neo-institutional theory argues that organizations face pressure to publicly adopt goals demanded constituencies in the external environment (e.g., move up in the rankings, increase socioeconomic and racial diversity) \citep{RN513,RN527}. However, organizations have scarce resources and cannot easily pursue goals that conflict with one another.  Rather than publicly rejecting a goal demanded by the external environment, organizations resolve conflicts between stated goals by substantively adopting some goals and symbolically adopting others.  Under substantive adoption, organizations allocate substantial resources towards achieving the goal.  Under symbolic adoption, organizations adopt policies and rhetoric that signal commitment to the goal, but do not allocate substantial resources to achieving the goal.  This theoretical perspective on organizational priorities is stated succinctly by the Joe Biden quote, ``don’t tell me what you value. Show me your budget and I’ll tell you what you value.''  

Off-campus recruiting visits by university admissions staff represent a substantial allocation of resources (e.g., staff salary and benefits, travel costs).  Therefore, we argue that comparing the characteristics of schools and communities that receive recruiting visits to those that do not can yield insights about university enrollment priorities.  By contrast, speeches and policy adoption (e.g., holistic admissions, ``outreach'' programs) shows which goals are publicly adopted \citep[e.g., ][]{RN4017}, but do not indicate which goals have been adopted substantively versus symbolically.


\begin{quote-block}  % quote block
    \lipsum[2]
\end{quote-block}

\lipsum[3]


\lipsum[1-2]

\begin{odd-block}  % text block for odd-numbered pages
  \vspace{-0.3cm}
  \subsection*{\color{colorTitle}{Key Points}}
  \vspace{-0.2cm}
  \lipsum[1-2]
\end{odd-block}

\lipsum[3]

\section*{Project overview\markboth{Project overview}{}}

This report presents descriptive results from a broader project that collects data on off-campus recruiting by colleges and universities. Many universities advertise off-campus recruiting events on their admissions websites (e.g. "coming to your area" links).  We used ``web-scraping'' to collect data on recruiting events.  We ``scraped'' web-pages containing recruiting event data once per week from 1/1/2017 to 12/31/2017, thereby capturing recruitment of spring juniors and fall seniors.

The data collection sample for the broader project was drawn from the population of public research-extensive universities (2000 Carnegie Classification). Out of all public research-extensive universities (N=102), the project collected data for those that posted off-campus recruiting events on their admissions websites (N=40). For each university in the project sample, we investigated the entire university website, searching for URLs that contained data on off-campus recruiting events. This process was conducted independently by two members of the research team to avoid missing any relevant URLs. Our programs also scraped data about participation in national college fairs from the National Association for College Admission Counseling (NACAC) website. We also collected data about participation in "group travel tours" from websites advertising joint recruiting events by multiple universities (e.g. Peach State Tour by Georgia State University, Georgia Tech, and The University of Georgia). Since URLs containing data on off-campus recruiting events often change (e.g., a university creates a new URL or changes the formatting of an existing URL),  we completed this investigation process for each university every 2 months and data collection scripts were updated accordingly.

\subsection*{Defining off-campus recruiting}

We categorize off-campus recruiting events based on event type, host, and location. Event type includes college fairs (in which multiple colleges attend), day-time high school visits, group travel visits, formal admissions interviews, admitted student events, and committed student events. Event hosts include paid staff, paid consultants (e.g. regional recruiter contracted by several institutions), alumni, and current students. Event locations include high schools, community colleges, hotels, conference/convention centers, and other public places (e.g., cafes).  

For the purpose of our research, we define off-campus recruiting events as those that focused on soliciting undergraduate admissions applications and were hosted by paid personnel or consultants at any off-campus location. This definition excludes admitted and committed student events, but includes guidance counselor events. Additionally, we excluded formal one-on-one formal interviews because these events are focused on determining the admission of one particular student rather than an open event soliciting applications from many prospective students. We excluded events hosted by alumni or student volunteers because research on organizational behavior finds that practices allocated to paid staff are better indicators of organizational priorities than those allocated to volunteers \citep{RN531}. 
%FOOTNOTE? ^[Or, event may be a virtual event (e.g., webinar, video call) with a target audience at a specific off-campus location (e.g., students from a particular high school)]

\subsection*{Data processing and data quality}

We took a multi-step approach to processing information scraped from admissions webpages. First, automated Python scripts scrape all text from admission webpages, storing the information as HTML text in a Structured Query Language (SQL) database on a remote server. Separate scripts parse the HTML text into tabular data (e.g., columns for event date, event time, school name, address). Third, we "geocode" recruiting events, converting limited location information (e.g., school name, city, state) into geographic coordinates. Geocoding scripts take location information, query the Google Maps Application Program Interface (API), and return more detailed geographic information for each event (e.g., latitude and longitude coordinates, county, city, state, full street address, zip code).

We conducted two additional data quality checks. First, we manually checked each scraped recruiting event, ensuring that event "type" (e.g., public high school visit) was correctly categorized and that each event was merged to the correct secondary data source (e.g., the correct NCES school ID). Second, we checked the completeness of web-scraped data by issuing public records requests for the list of all off-campus recruiting events and then comparing the two data sources.
%SAY THAT WE COULD ONLY DO THIS FOR X UNIVERSITIES?

\subsection*{Analysis sample}

The analysis sample for this manuscript consists of 15 public research universities. These cases were selected from the larger project sample and selected based on "completeness" of recruiting event data posted on admissions websites.  Based on prior research and conversations with admissions professionals, nearly all colleges and universities convene three broad types of off-campus recruiting events: (1) receptions/college fairs at hotels and convention centers; (2) evening college fairs at local high schools; and (3) day-time visits at local high schools. However, some institutions we collected data from did not post all three types of recruiting events. Of the 40 public research universities we collected data on, these 15 universities posted all three broad types of off-campus recruiting events on their website. TABLE X [ADD] shows how the 15 universities in our sample compare to the population of public research universities.

\section*{Results\markboth{Results}{}}

\subsection*{University of Georgia}
\lipsum[4]
% Params: univ_id (which is also the name of the folder inside images/ folder, where the images are located), caption, fig reference
\addFigureSet{139959}{University of Georgia result set.}{uga}
\lipsum[5]

\subsection*{University of California, Berkeley}
\lipsum[4]
\addFigureSet{110635}{UC Berkeley result set.}{ucberkeley}
\lipsum[4]

\subsection*{University of Pittsburgh}
\lipsum[4]
\addFigureSet{215293}{University of Pittsburgh result set.}{upitt}
\lipsum[5]

\subsection*{Stony Brook}
\lipsum[4]
\addFigureSet{196097}{Stony Brook result set.}{stonybrook}
\lipsum[5]

\section*{Cross-University Results\markboth{Cross-University Results}{}}

\lipsum[1]

\addFigureCompare{Comparison Across Universities\hspace{1cm}}{compare}

\lipsum[6]

\section*{Conclusions\markboth{Conclusions}{}}

\lipsum[1]

\begin{itemize}  % List
	\item Un
	\begin{itemize}
		\item Lorem ipsum dolor sit amet, consectetuer adipiscing elit.
		\item Cras nec ante. Pellentesque a nulla.
	\end{itemize}
	\item Deux
	\begin{itemize}
		\item Nam dui ligula, fringilla a, euismod sodales, sollicitudin vel, wisi.
	\end{itemize}
	\item Ttrois
	\begin{itemize}
		\item Duis aute irure dolor in reprehenderit in voluptate velit esse cillum dolore eu fugiat nulla pariatur.
	\end{itemize}
	
\end{itemize} 

\lipsum[4]

\newpage

\bibliographystyle{apacite}
\def\bibfont{\small}

\bibliography{spencer-bib}


\newpage

% Endnotes ~~~~

\section*{Endnotes\markboth{Endnotes}{}}

TBD

% Final Page ~~~~

\newpage
\pagecolor{LightBlue}
\begin{titlepage}
    \color{LightBlue}{x}
\end{titlepage}

\end{document}
